\begin{figure}[H]
\begin{center}
\def\layersep{1.1 in}
{
\begin{tikzpicture}[shorten >=1pt,->,draw=black!50, x = 0.2 in, y = 0.2 in,  node distance=\layersep]
%\draw[draw=none, use as bounding box](-4.0,0) rectangle (%-10);
%\node[anchor=base,-] at (-2.25,-5.5) {\chemfig[][scale=1]{{\color{darkgreen}L_{eq}}>[:30]{\color{orange}M}(-[90,,,,thick]{{\color{green}L_{ax}}})(-[:-90,,,,thick]{{\color{green}L_{ax}}})
%(<:[:150]{\color{darkgreen}{L_{eq}}})
%(<[:-30]{\color{darkgreen}{L_{eq}}})
%<:[:30]{\color{darkgreen}{L_{eq}}}}};

    \tikzstyle{every pin edge}=[<-,shorten <=1pt,thick]
    \tikzstyle{neuron}=[circle,fill=black!25,minimum size=0.3in ,inner sep=0pt, color=black, draw]
    \tikzstyle{input neuron}=[neuron, fill=green!50!blue!50];
    \tikzstyle{output neuron}=[neuron, fill=green!80!darkgreen!80!olive!50!blue!50!];
    \tikzstyle{hidden neuronone}=[neuron, fill=green!50!darkgreen!50!olive!50];
    \tikzstyle{hidden neurontwo}=[neuron, fill=green!80!darkgreen!80!olive!80!blue!10!];
    \tikzstyle{annot} = [text width=2em, text centered]

    % Draw the input layer nodes
    \foreach \name / \y in {3}
    % This is the same as writing \foreach \name / \y in {1/1,2/2,3/3,4/4}
%        \node[input neuron, pin=left:\footnotesize Input] (I-\name) at (0,-\y) {$ $};
        \node[input neuron,fill=green!100 ] (I-\name) at (0,-\y) {$ $};
            \foreach \name / \y in {5}
    % This is the same as writing \foreach \name / \y in {1/1,2/2,3/3,4/4}
%        \node[input neuron, pin=left:\footnotesize Input] (I-\name) at (0,-\y) {$ $};
        \node[input neuron,fill=darkgreen!100] (I-\name) at (0,-\y) {$ $};
            \foreach \name / \y in {7}
    % This is the same as writing \foreach \name / \y in {1/1,2/2,3/3,4/4}
%        \node[input neuron, pin=left:\footnotesize Input] (I-\name) at (0,-\y) {$ $};
        \node[input neuron,fill=olive] (I-\name) at (0,-\y) {$ $};
    % Draw the hidden layer nodes
    \foreach \name / \y in {2,4,6,8}
        \path[yshift=0.5]
            node[hidden neuronone] (H-\name) at (\layersep,-\y) {\small $ $};
    \foreach \name / \y in {2,4,6,8}
        \path[yshift=0.5]
            node[hidden neurontwo] (H2-\name) at (2*\layersep,-\y) {\small $ $};
    % Draw the output layer node
%    \node[output neuron,pin={[pin edge={->}]right:\footnotesize Output}, right of=H2-7] (O) {$ $};
         \path[yshift=0.5] node[output neuron] (O) at (3*\layersep,-5) {$ $};
%    % Connect every node in the input layer with every node in the
%    % hidden layer.

    \foreach \source in {3}
        \foreach \dest in {2,4,6,8}
            \path[very thick,green] (I-\source) edge node[near start,font=\scriptsize,sloped] {} (H-\dest) ;

    \foreach \source in {5}
        \foreach \dest in {2,4,6,8}
            \path[very thick,darkgreen] (I-\source) edge node[near start,font=\scriptsize,sloped] {} (H-\dest) ;

    \foreach \source in {7}
        \foreach \dest in {2,4,6,8}
            \path[very thick,orange] (I-\source) edge node[near start,font=\scriptsize,sloped] {} (H-\dest) ;
            
            
                \foreach \source in {2,4,6,8}
        \foreach \dest in {2,4,6,8}
            \path[very thick] (H-\source) edge node[near start,font=\scriptsize,sloped] {} (H2-\dest) ;
    \foreach \source in {2,4,6,8}
           \path[very thick] (H2-\source) edge node[near end,font=\scriptsize,sloped] {}(O) ;
%    % Connect every node in the hidden layer with the output layer
%    \foreach \source in {1,...,3}
%        \path (H-\source) edge node[midway,font=\scriptsize,sloped] {$w_{\source 1}$}(O);
%    % Annotate the layers
    \node[annot,above of=H-2, node distance= 1cm] (hl) {\footnotesize Hidden layer};
    \node[annot,left of=hl] (hi) {\footnotesize Input layer};
    \node[annot,right of=hl] (hl2) {\footnotesize Hidden layer};
    \node[annot,right of=hl2] {\footnotesize Output layer};
    \path[very thick, draw, green] (-4.0,-3) -- node[above, sloped] {} (I-3.west) ;
        \path[very thick, draw, darkgreen] (-4.0,-5) -- node[above, sloped] {} (I-5.west) ;
    \path[very thick, draw, olive] (-4.0,-7) -- node[above, sloped] {} (I-7.west) ;
	\node[right=1 cm of O.east] (endp) {}; 
   \path[very thick, draw, blue!50!green] (O.east) -- (endp);
\end{tikzpicture}}\end{center}
\caption{Schematic representation of basic structure of multilevel neural network model. Inputs (green) are passed to nonlinear activation functions, visualized as nodes, which comprise the hidden layers of the network. The weighting of each term in every linear combination, visualized as arrows, are a tunable parameters determined during training. Not visualized are additive bias term included at every node. \label{Fig:anneq}}
\end{figure}

