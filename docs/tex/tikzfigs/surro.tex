
\begin{figure}[H]
%\noindent\resizebox{15cm}{!}{
\begin{tikzpicture}
\node (A) at (3.45,3.75) {{\chemfig[][scale=1]{{\color{darkgreen}L_{1}}>[:30]{\color{orange}M}(-[90,,,,thick]{{\color{blue}L_{2}}})(-[:-90,,,,thick]{{\color{green}L_{3}}})
(<:[:150]{\color{darkgreen}{L_{1}}})
(<[:-30]{\color{darkgreen}{L_{1}}})
<:[:30]{\color{darkgreen}{L_{1}}}}}};
\node (ps) at (13,0){};
\node (psy) at (13,4){};
\path[line,very thick] (psy) -- (ps.center);
\node[circle, fill=black,minimum width =0.05cm,label=below:{$\mathbf{x}_i$}] (x) at (8.2,2.8) {};
\node[circle, fill=black,minimum width =0.05cm,label=left:{$\mathbf{x}_{i+1}$}] (x2) at (6.3,2.6) {};
\node[circle, fill=black,minimum width =0.05cm,label=right:{$\mathbf{x}_{j}$}] (x3) at (5.5,0.75) {};
\node[circle, fill=red,minimum width =0.05cm,label=right:{$\mathbf{y}_{i}$}] (y) at (13,3) {};
\node[circle, fill=blue,minimum width =0.05cm,label=right:{$\mathbf{y}_{i+1}$}] (y2) at (13,2) {};
\node[circle, fill=darkgreen,minimum width =0.05cm,label=right:{$\mathbf{y}_{j}$}] (y3) at (13,0.75) {};
\path[draw, thick,->,red] (A) edge[bend left] node[below] {$T(c_i)$} (x);
\path[draw, thick,->] (x) edge[bend left] node[below] {$\tilde{f}(c_i)$} (y);
%% path iin DS
\path[draw,dashed,gray,thick,->] (x) -- (x2);
\path[draw,dashed,gray,thick,->] (x2) -- (x3);
\node (ds) at (5,0){};
\node (dsy) at (5,4){};
\node (dsx) at (9,0){};
\path[line,very thick] (ds.center) -- (dsx);
\path[line,very thick] (ds.center) -- (dsy);
\node (lab1) at (7,-0.5){\textbf{Descriptor Space $X\subset \mathbb{R}^{d}$}};
\node (lab1) at (13,-0.5){\textbf{Estimated Objective $\mathbb{R}$}};
\node (lab1) at (1,-0.5){\textbf{Chemical Space $C_f$}};
\node (is2) at (12,2){};
\node (is1) at (9,2){};
\node (is3) at (9,1){};
\node (is4) at (12,1){};
\path[draw, thick,->] (is1) edge[bend left] node[below] {} (is2);
\path[draw, thick,->] (is4) edge[bend left] node[below] {} (is3);
\node (lab3) at (10.5,1.5){ $\min \tilde{f}$ on $X$};
\foreach \x in {0,1,2}
		\foreach \y in {1,2,3}{
		\node[rectangle,fill=teal,minimum width = 0.05cm] (place\x\y) at (\x,\y){};}
\node[draw,circle,very thick,red,minimum width = 0.5cm,label=right:{\color{red}$c_i$}] (ci) at (1,3){};
\node[draw,circle,very thick,blue,minimum width = 0.5cm,label=right:{\color{blue}$c_j$}] (cj) at (2,1){};
\node[draw,circle,very thick,red,minimum width = 2.75cm] (cic) at (3.45,3.75){};
\path[draw,red,dashed,very thick] (ci.north) edge node[below] {} (cic.north west);
\path[draw,red,dashed,very thick] (ci.south) edge node[below] {} (cic.south);
\path[draw, thick,->,blue] (x3) edge[bend left] node[above] {$T^{-1}(x_{j})$} (cj);
\end{tikzpicture}
%}
\caption{General surrogate model scheme for optimization, showing how to select a new point $c_j$ in chemical space from existing point $c_i$ (illustrated), by projection ($T$) into the descriptor space $X$, optimization using $\tilde{f}$, and then reconstruction, $T^{-1}\mathbf{x}_j$ . $\tilde{f}$ is a surrogate model parameterized on knowledge of the expensive objective (e.g. DFT) at previous points including $c_i$. \label{Fig:surro}}
\end{figure}